\section{Task definition}
\label{task-definition}

The first sub-task of SemEval-2020 Task 3 is to predict the direction and magnitude of
the change in the human judgment of similarity of the same pair of target words in two
different contexts.
The task is unsupervised: the submissions were evaluated on the CoSimLex dataset
\parencite[39-42]{Armendariz2020} but only a minimal `practice kit' of fewer than ten
instances was provided in advance.
CoSimLex is an extension of SimLex-999 \parencite{Hill2015} that consists of pairs of
target words and their contexts in four languages: English ($n = 340$), Finnish ($n =
  24$), Croatian ($n = 112$), and Slovene ($n = 111$).

The score for the first sub-task was computed by the uncentered (zero-mean) Pearson
correlation coefficient between the predicted changes in similarity and the human
judgments represented in the CoSimLex dataset \parencite[42]{Armendariz2020}.
This metric is equivalent to the cosine similarity between the two vectors of results:
\begin{equation}
  \text{score}(\vec{\hat{y}}, \vec{y})
  = \frac{\sum_{i = 1}^{n} \hat{y}_i y_i}{\left( \sum_{i = 1}^{n} \hat{y}_i^2 \right)\left( \sum_{i = 1}^{n} y_i^2 \right)}
  = \frac{\vec{\hat{y}} \cdot \vec{y}}{\norm{\vec{\hat{y}}} \norm{\vec{y}}}
\end{equation}
Notably, this metric is invariant with respect to multiplication by a scalar quantity,
so the results of composing an equal number of representations by addition or the
arithmetic mean are equal.
Hence, I only investigated addition among the two (\cref{sec:methodology}).
