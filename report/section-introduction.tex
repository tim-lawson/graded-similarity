\section{Introduction}
\label{sec:introduction}

In his \emph{Foundations of Arithmetic}, Frege promises ``never to ask for the meaning
of a word in isolation, but only in the context of a proposition''
\parencite*[xvii]{Frege1980}.
This `context principle' is intuitive: words are frequently polysemous, or assume
different connotations and emphasis within different expressions
\parencite[2-3]{Armendariz2020}.
Historically, however, context-dependence has been a problem for distributional meaning
representations.
Founded on the distributional hypothesis \parencite[e.g.,][142-143]{Turney2010}, both
count-based and predictive models of word meaning\footnote{ This terminological
  distinction is due to \textcite{Baroni2014a}.
} originally
produced a single representation for each word in the model's vocabulary.
One of these \emph{static} representations must, therefore, encode all of a word's
senses and connotations, which may obstruct its use in modelling context-dependent
phenomena.

Prior to the widespread availability of pre-trained language models, this problem was
generally addressed by one of two approaches: firstly, by producing a representation
for each sense of a target word and disambiguating between them in the given context
(word-sense disambiguation); or secondly, by composing the representation of the target
word with the representations of the words in its context (contextualization).
These approaches have been largely overshadowed by the advent of model architectures
that take sequences as inputs and naturally produce \emph{contextual} representations
of the items in the sequence, such as Transformers \parencite{Vaswani2017}.

To my knowledge, however, there has been scant direct comparison of the performance of
these contextual representations with the application of prior methods of
contextualization to static representations (\cref{sec:related-work}).
SemEval-2020 Task 3, ``Graded Word Similarity in Context'' \parencite{Armendariz2020a},
presents an opportunity to make such a comparison.
Briefly, the task is to predict the human judgment of similarity of the same pair of
words in two different contexts.
I elected to focus on the first sub-task, which is to predict the \emph{change} in
similarity, rather than the absolute similarity in each context
(\cref{task-definition}).
Specifically, I evaluated the results obtained by computing the cosine similarity
between static and contextual embeddings and the composition of these embeddings within
a fixed-size context window (\cref{sec:methodology}).
