\section{Results}
\label{sec:results}

\subsection{Hyperparameter search}
\label{sec:hyperparameter-search}

In \cref{sec:cost-benefit,sec:language-specificity,sec:significance}, I present the
results for different models on the \emph{evaluation} dataset.
However, it would not have been possible or legitimate as a task submission to optimize
hyperparameters on the evaluation dataset.
Therefore, I also carried out this procedure on the `practice kit' dataset
(\cref{sec:task-definition}) to select a candidate model for each language and kind of
embedding.
This data was not provided for Finnish, so it was excluded from the analysis.
In comparison to addition, the scores for the other composition operations varied more
widely with respect to the window size
(\cref{chart:score-window-static-practice,chart:score-window-contextual-practice,chart:score-window-pooled-practice}).
Hence, I excluded them before selecting candidate models.
The candidate models and their scores on the two datasets are listed in
\cref{table:practice-best-score}.
As expected, the scores on the `practice kit' dataset of fewer instances are generally
higher than those on the evaluation dataset.
In some cases, the scores on the evaluation dataset are close to the maxima in
\cref{table:evaluation-best-score}.
Generally, the benefit of additional `training' data is evident.

\import{}{figure-score-practice.tex}

\import{}{table-practice-best-score.tex}

\subsection{Cost-benefit analysis of contextual embeddings}
\label{sec:cost-benefit}

\import{}{figure-prediction.tex}

In the main, greater scores were achieved with contextual and pooled embeddings than
with static embeddings (\cref{table:practice-best-score,table:evaluation-best-score}).
However, static embeddings make up a small fraction of the size of a contextual
language model.
For example, BERT's vocabulary size is approximately $30000$, the dimensions of the
\texttt{bert-base} and \texttt{bert-large} variants' hidden-states are $768$ and
$1024$, and their total parameters are $110$M and $340$M respectively
\parencite[4173-4174]{Devlin2019}.
Static embeddings thus make up approximately $21$\% and $9$\% of the total parameters.
It is also much faster to compute a contextualized representation from static
embeddings than to run inference on a language model.
For a naïve implementation of the procedure described in \cref{sec:methodology}, the
approximate time taken to compute the change in similarity between two words in context
is shown in \cref{chart:score-time}.
It is notably greater for contextual embeddings.
The right-most cluster is due to the \texttt{large} model variants.

\import{}{figure-score-time.tex}

\subsection{Language-specificity of window size}
\label{sec:language-specificity}

\import{}{table-evaluation-best-score.tex}

Generally, I found that the scores obtained by all three types of embeddings were
maximized by a non-zero context-window size
(\cref{table:practice-best-score,table:evaluation-best-score}).
The influence of the window size is intuitive in the case of static embeddings.
Without a context window, the representations of a target word only differ between
expressions if the word is represented by different sub-word tokens in the different
expressions.
A similar argument applies to contextual embeddings, in that a target word may be
represented by multiple sub-word tokens.
For the composition operation of addition, the scores against window size for each
language, kind of embedding, and model are shown in
\cref{chart:score-window-static,chart:score-window-contextual,chart:score-window-pooled}.

\textcite[3]{Virtanen2019} have noted that, for a random sample of 1\% of the relevant
Wikipedia dataset, the number of sub-word tokens used to represent a word by a
multilingual BERT model is greater for Finnish ($1.97$) than for English ($1.16$).
This is attributed to the morphological complexity of Finnish and its comparatively
small fraction of the multilingual model's vocabulary.
Accordingly, I found that Finnish-specific models generally outperformed multilingual
ones and that scores varied more widely with window size for Finnish than the other
languages.

\import{}{figure-score-window.tex}

\subsection{Significance tests}
\label{sec:significance}

Finally, to quantify the significance of the differences between the scores obtained
with different kinds of embeddings, I conducted paired $t$-tests of the scores of the
best models (\cref{table:evaluation-best-score}) over ten random samples of 90\% of the
evaluation dataset.
For each pair of models, the null hypothesis is that the two sets of scores have the
same mean, i.e., there is no significant difference between the mean scores.

In all cases, the null hypothesis was rejected at the $p = 0.05$ significance level.
Contextual and pooled embeddings significantly outperformed static embeddings for each
language, but whether contextual or pooled embeddings were preferred depended on the
language (\cref{table:significance}).
Hence, the results substantiate the assertion that contextual embeddings are more
effective than static embeddings for this task (\cref{sec:cost-benefit}).
Notably, the mean scores for Finnish differed less from each other than the other
languages.

\import{}{table-significance.tex}