\section{Results}
\label{sec:results}

\subsection{Cost-benefit analysis of contextual embeddings}
\label{sec:cost-benefit}

\import{.}{figure-chart-score-time}

\todonum{Write this section.}

\subsection{Language-specificity of window-size effects}

Generally, I found that the scores obtained by both static- and contextual-embedding
models were maximised by a non-zero context-window size.
Due to the computational expense of exhaustively searching the possible window sizes,
I applied a heuristic to constrain the search space.
A naïve estimation of the average number of words in each context, i.e., segmenting on
whitespace, gave a result of between $40$ and $60$ for the different languages.
Therefore, for the static-embedding models, I chose $50$ as an upper bound on the
window size.
The motivation to choose a smaller maximum window size for contextual-embedding models
was similarly economical (Section~\ref{sec:cost-benefit}); however, as the window size
approaches the length of the sequence, one would expect a combination of token
representations to be superseded by the sequence-level representation of the model,
e.g., the special \texttt{CLS} token of BERT variants \parencite[4174]{Devlin2019}.
These heuristics were largely vindicated by the results of the evaluation, which
demonstrated that the scores decrease as the window size approaches the maximum.

The influence of the window size is intuitive in the case of static embeddings.
Without a context window, the representations of a target word only differ between
expressions if the word is represented by different sub-word tokens in the different
expressions.
A similar argument applies to contextual embeddings, in that a target word may be
represented by multiple sub-word tokens.
The window sizes that maximise the score for each language and model are given in
Table~\ref{table:best-window-size}.

\import{.}{figure-table-best-window-size}
\import{.}{figure-chart-score-window}