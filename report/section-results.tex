\section{Results}
\label{sec:results}

\subsection{Hyperparameter search}

In the following sections, I present the results achieved on the evaluation dataset.
However, by comparing the scores achieved with different kinds of embeddings, window
sizes, etc., I have effectively performed hyperparameter search on the evaluation
dataset, which would not have been possible or legitimate as a task submission.
Therefore, I also explicitly implemented a hyperparameter search procedure on the
`practice kit' to select a candidate model for each language and computed the scores of
these models on the evaluation dataset.
\\
\todonum{Add best `practice kit' candidates and results on evaluation data.}

\subsection{Cost-benefit analysis of contextual embeddings}
\label{sec:cost-benefit}

\import{}{figure-table-best-score.tex}

In the main, greater scores were achieved with contextual and pooled embeddings than
with static embeddings.
However, static embeddings make up a small fraction of the size of a contextual
language model, and it is much faster to compute a contextualized representation from
static embeddings than to run inference on a language model.
The second of these advantages is shown in \cref{chart:score-time}.
For a naïve implementation of the procedure described in \cref{sec:methodology}, the
approximate time taken to compute the change in similarity between two words in
context, i.e., the total time divided by the number of instances, is notably greater
for contextual embeddings than for static embeddings.
The right-most cluster is due to the \texttt{large} model variants.
\\
\todonum{Plot the gold-standard values against the predictions for the best model
  for each language to visualize the Pearson correlation coefficient.}

\import{}{figure-chart-score-time.tex}

\subsection{Language-specificity of window-size effects}
\label{sec:language-specificity}

Generally, I found that the scores obtained by all three types of embeddings were
maximized by a non-zero context-window size.
The influence of the window size is intuitive in the case of static embeddings.
Without a context window, the representations of a target word only differ between
expressions if the word is represented by different sub-word tokens in the different
expressions.
A similar argument applies to contextual embeddings, in that a target word may be
represented by multiple sub-word tokens.
For the additive composition operation, the scores against window size for each
language and model are given in
\cref{chart:score-window-static,chart:score-window-contextual,chart:score-window-pooled}.

\textcite[3]{Virtanen2019} noted that, for a random sample of 1\% of the relevant
Wikipedia dataset, the number of sub-word tokens used to represent a word by a
multilingual BERT model is greater for Finnish ($1.97$) than for English ($1.16$).
This is attributed to the morphological complexity of Finnish and its comparatively
small fraction of the multilingual model's vocabulary.
Accordingly, I found that Finnish-specific models generally outperformed multilingual
ones and that the score varied more widely with window size for Finnish than the other
languages.

% \import{}{figure-table-best-window-size}
\import{}{figure-chart-score-window.tex}

\subsection{Significance tests}

Finally, to quantify the significance of the differences between the scores obtained
with different kinds of embeddings, I computed paired $t$-tests of the scores of the
best models in each class over the same ten random samples of 90\% of the evaluation
dataset.
For each pair of model classes, the null hypothesis is that the two sets of scores have
the same mean, i.e., there is no significant difference between the mean score obtained
by the two classes.
\\
\todonum{Add significance test results.}