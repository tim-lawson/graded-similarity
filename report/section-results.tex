\section{Results}
\label{sec:results}

\subsection{Hyperparameter search}
\label{sec:hyperparameter-search}

\newcommand{\plotscorepractice}[2]{
  \foreach \modelname in #2
    {
      \addplot+[
        sharp plot,
        mark = none,
        line width = 0.8pt,
        discard if not = {model_name}{\modelname},
      ] table [
          col sep=comma,
          x = window,
          y = score,
        ]{#1};
    }
}

\begin{figure}
  \centering
  \begin{tikzpicture}
    \begin{groupplot}[
        group style={
            group size=3 by 3,
            xlabels at=edge bottom,
            ylabels at=edge left,
            vertical sep=0.2in,
            horizontal sep=0.2in,
          },
        width=2.25in,
        height=1.6in,
        xlabel=Window size,
        ylabel=Score,
        ymin=-0.5,
        ymax=1,
        xtick={0,10,...,50},
        ytick={-0.5,0,0.5,1.0},
        xticklabel=\empty,
        yticklabel=\empty,
        cycle list/Paired,
    ]
      \nextgroupplot[title=Addition, ylabel={English ($n\,{=}\,10$)}, yticklabels={-0.5,0,0.5,1.0}]
      \plotscorepractice{../results/practice/model=static_language=en_window=0-50_operation=sum_similarity=cosine.csv}{\modelnameen}
      \nextgroupplot[title=Multiplication]
      \plotscorepractice{../results/practice/model=static_language=en_window=0-50_operation=prod_similarity=cosine.csv}{\modelnameen}
      \nextgroupplot[title=Concatenation]
      \plotscorepractice{../results/practice/model=static_language=en_window=0-50_operation=concat_similarity=cosine.csv}{\modelnameen}
      \nextgroupplot[ylabel={Croatian ($n\,{=}\,5$)}, yticklabels={-0.5,0,0.5,1.0}]
      \plotscorepractice{../results/practice/model=static_language=hr_window=0-50_operation=sum_similarity=cosine.csv}{\modelnamehr}
      \nextgroupplot
      \plotscorepractice{../results/practice/model=static_language=hr_window=0-50_operation=prod_similarity=cosine.csv}{\modelnamehr}
      \nextgroupplot
      \plotscorepractice{../results/practice/model=static_language=hr_window=0-50_operation=concat_similarity=cosine.csv}{\modelnamehr}
      \nextgroupplot[ylabel={Slovene ($n\,{=}\,5$)}, xticklabels={0,10,...,50}, yticklabels={-0.5,0,0.5,1.0}]
      \plotscorepractice{../results/practice/model=static_language=sl_window=0-50_operation=sum_similarity=cosine.csv}{\modelnamesl}
      \nextgroupplot[xticklabels={0,10,...,50}]
      \plotscorepractice{../results/practice/model=static_language=sl_window=0-50_operation=prod_similarity=cosine.csv}{\modelnamesl}
      \nextgroupplot[xticklabels={0,10,...,50}]
      \plotscorepractice{../results/practice/model=static_language=sl_window=0-50_operation=concat_similarity=cosine.csv}{\modelnamesl}
    \end{groupplot}
  \end{tikzpicture}
  \caption{The score on the `practice kit' dataset against window size for \emph{static}
    embedding models.
    The model-name legends are omitted for brevity but match
    \cref{chart:score-window-static}.
  }
  \label{chart:score-window-static-practice}
\end{figure}

\begin{figure}
  \centering
  \begin{tikzpicture}
    \begin{groupplot}[
        group style={
            group size=3 by 3,
            xlabels at=edge bottom,
            ylabels at=edge left,
            vertical sep=0.2in,
            horizontal sep=0.2in,
          },
          width=2.25in,
          height=1.6in,
          xlabel=Window size,
          ylabel=Score,
          ymin=-0.5,
          ymax=1,
          xtick={0,2,4,6,8,10},
          ytick={-0.5,0,0.5,1.0},
          xticklabel=\empty,
          yticklabel=\empty,
          cycle list/Paired,
    ]
      \nextgroupplot[title=Addition, ylabel={English ($n\,{=}\,10$)}, yticklabels={-0.5,0,0.5,1.0}]
      \plotscorepractice{../results/practice/model=contextual_language=en_window=0-10_operation=sum_similarity=cosine.csv}{\modelnameen}
      \nextgroupplot[title=Multiplication]
      \plotscorepractice{../results/practice/model=contextual_language=en_window=0-10_operation=prod_similarity=cosine.csv}{\modelnameen}
      \nextgroupplot[title=Concatenation]
      \plotscorepractice{../results/practice/model=contextual_language=en_window=0-10_operation=concat_similarity=cosine.csv}{\modelnameen}
      \nextgroupplot[ylabel={Croatian ($n\,{=}\,5$)}, yticklabels={-0.5,0,0.5,1.0}]
      \plotscorepractice{../results/practice/model=contextual_language=hr_window=0-10_operation=sum_similarity=cosine.csv}{\modelnamehr}
      \nextgroupplot
      \plotscorepractice{../results/practice/model=contextual_language=hr_window=0-10_operation=prod_similarity=cosine.csv}{\modelnamehr}
      \nextgroupplot
      \plotscorepractice{../results/practice/model=contextual_language=hr_window=0-10_operation=concat_similarity=cosine.csv}{\modelnamehr}
      \nextgroupplot[ylabel={Slovene ($n\,{=}\,5$)}, xticklabels={0,2,4,6,8,10}, yticklabels={-0.5,0,0.5,1.0}]
      \plotscorepractice{../results/practice/model=contextual_language=sl_window=0-10_operation=sum_similarity=cosine.csv}{\modelnamesl}
      \nextgroupplot[xticklabels={0,2,4,6,8,10}]
      \plotscorepractice{../results/practice/model=contextual_language=sl_window=0-10_operation=prod_similarity=cosine.csv}{\modelnamesl}
      \nextgroupplot[xticklabels={0,2,4,6,8,10}]
      \plotscorepractice{../results/practice/model=contextual_language=sl_window=0-10_operation=concat_similarity=cosine.csv}{\modelnamesl}
    \end{groupplot}
  \end{tikzpicture}
  \caption{The score on the `practice kit' dataset against window size for
    \emph{contextual} embedding models.
    The model-name legends are omitted for brevity but match
    \cref{chart:score-window-contextual}.
  }
  \label{chart:score-window-contextual-practice}
\end{figure}

\begin{figure}
  \centering
  \begin{tikzpicture}
    \begin{groupplot}[
      group style={
          group size=3 by 3,
          xlabels at=edge bottom,
          ylabels at=edge left,
          vertical sep=0.2in,
          horizontal sep=0.2in,
        },
      width=2.25in,
      height=1.6in,
      xlabel=Window size,
      ylabel=Score,
      ymin=-0.5,
      ymax=1,
      xtick={0,2,4,6,8,10},
      ytick={-0.5,0,0.5,1.0},
      xticklabel=\empty,
      yticklabel=\empty,
      cycle list/Paired,
    ]
      \nextgroupplot[title=Addition, ylabel={English ($n\,{=}\,10$)}, yticklabels={-0.5,0,0.5,1.0}]
      \plotscorepractice{../results/practice/model=pooled_language=en_window=0-10_operation=sum_similarity=cosine.csv}{\modelnameen}
      \nextgroupplot[title=Multiplication]
      \plotscorepractice{../results/practice/model=pooled_language=en_window=0-10_operation=prod_similarity=cosine.csv}{\modelnameen}
      \nextgroupplot[title=Concatenation]
      \plotscorepractice{../results/practice/model=pooled_language=en_window=0-10_operation=concat_similarity=cosine.csv}{\modelnameen}
      \nextgroupplot[ylabel={Croatian ($n\,{=}\,5$)}, yticklabels={-0.5,0,0.5,1.0}]
      \plotscorepractice{../results/practice/model=pooled_language=hr_window=0-10_operation=sum_similarity=cosine.csv}{\modelnamehr}
      \nextgroupplot
      \plotscorepractice{../results/practice/model=pooled_language=hr_window=0-10_operation=prod_similarity=cosine.csv}{\modelnamehr}
      \nextgroupplot
      \plotscorepractice{../results/practice/model=pooled_language=hr_window=0-10_operation=concat_similarity=cosine.csv}{\modelnamehr}
      \nextgroupplot[ylabel={Slovene ($n\,{=}\,5$)}, xticklabels={0,2,4,6,8,10}, yticklabels={-0.5,0,0.5,1.0}]
      \plotscorepractice{../results/practice/model=pooled_language=sl_window=0-10_operation=sum_similarity=cosine.csv}{\modelnamesl}
      \nextgroupplot[xticklabels={0,2,4,6,8,10}]
      \plotscorepractice{../results/practice/model=pooled_language=sl_window=0-10_operation=prod_similarity=cosine.csv}{\modelnamesl}
      \nextgroupplot[xticklabels={0,2,4,6,8,10}]
      \plotscorepractice{../results/practice/model=pooled_language=sl_window=0-10_operation=concat_similarity=cosine.csv}{\modelnamesl}
    \end{groupplot}
  \end{tikzpicture}
  \caption{The score on the `practice kit' dataset against window size for \emph{pooled}
    embedding models.
    The model-name legends are omitted for brevity but match
    \cref{chart:score-window-pooled}.
  }
  \label{chart:score-window-pooled-practice}
\end{figure}


In \cref{sec:cost-benefit,sec:language-specificity}, I present the results for
different models on the \emph{evaluation} dataset.
However, it would not have been possible or legitimate as a task submission to optimize
hyperparameters on the evaluation dataset.
Therefore, I also optimized them on the `practice kit' dataset
(\cref{sec:task-definition}) to select a candidate model for each language and kind of
embedding.
This data was not provided for Finnish, so it was excluded from the analysis.
In comparison to addition, the scores for the other composition operations varied more
widely with respect to the window size
(\cref{chart:score-window-static-practice,chart:score-window-contextual-practice,chart:score-window-pooled-practice}).
Hence, I excluded them before selecting candidate models.
The models and their scores on the two datasets are listed in
\cref{table:practice-best-score}.
As expected, the scores on the `practice kit' dataset of fewer instances are generally
higher than those on the evaluation dataset.
In some cases, the scores on the evaluation dataset are close to the maxima in
\cref{table:evaluation-best-score}.
Generally, the benefit of additional `training' data is evident.

\pgfplotstableset{
  columns/model_name/.style={
      column name=Model name,
      column type={L{2.25in}},
      string type,
      postproc cell content/.append style={
          @cell content={\texttt{##1}}}},
  columns/scorepractice/.style={
      column name={Practice},
      column type={R{0.5in}},
      fixed,
      fixed zerofill,
      precision=3},
  columns/scoreevaluation/.style={
      column name={Evaluation},
      column type={R{0.7in}},
      fixed,
      fixed zerofill,
      precision=3},
}

\begin{table}
  \centering
  \begin{subfigure}[b]{\textwidth}
    \centering
    \pgfplotstabletypeset[columns={
          language,
          model_name,
          window,
          scorepractice,
          scoreevaluation}]{../results/best/both_best_static.csv}
    \caption{Static}
  \end{subfigure}
  \par\bigskip
  \begin{subfigure}[b]{\textwidth}
    \centering
    \pgfplotstabletypeset[columns={
          language,
          model_name,
          window,
          scorepractice,
          scoreevaluation}]{../results/best/both_best_contextual.csv}
    \caption{Contextual}
  \end{subfigure}
  \par\bigskip
  \begin{subfigure}[b]{\textwidth}
    \centering
    \pgfplotstabletypeset[columns={
          language,
          model_name,
          window,
          scorepractice,
          scoreevaluation}]{../results/best/both_best_pooled.csv}
    \caption{Pooled}
  \end{subfigure}
  \caption{The best scores on the \emph{`practice kit'} dataset for each kind of
    embedding, and the corresponding scores on the evaluation dataset.
    The results were limited to the composition operation of addition due to the
    variability of the scores with multiplication and concatenation
    (\cref{chart:score-window-static-practice,chart:score-window-contextual-practice,chart:score-window-pooled-practice}).
  }
  \label{table:practice-best-score}
\end{table}


\subsection{Cost-benefit analysis of contextual embeddings}
\label{sec:cost-benefit}

\newcommand{\plotpredictedactual}[2]{
  \addplot[
    only marks,
    mark size = 0.8pt,
    color = #2,
  ] table [
      col sep=comma,
      x = predicted,
      y = actual,
    ]{#1};
}

\begin{figure}
  \centering
  \begin{tikzpicture}
    \begin{groupplot}[
        group style = {
            group size = 2 by 2,
            ylabels at = edge left,
            xlabels at = edge bottom,
            vertical sep = 0.7in,
            horizontal sep = 0.5in,
          },
        width = 3in,
        height = 2in,
        xlabel = Predicted,
        ylabel = Actual,
        cycle list/Paired,
        legend pos = south east,
        legend style = {
            draw = none,
            font = \small,
          },
        legend cell align={left},
      ]
      \nextgroupplot[title = English ($n\,{=}\,340$)]
      \plotpredictedactual{../results/predictions/embedding=contextual_model_name=bert-base-uncased_language=en_window=1_operation=sum_similarity=cosine.csv}{Paired-A}
      \node[anchor = north west] at (rel axis cs: 0.05, 0.95) {$r = 0.660$};
      \nextgroupplot[title = Finnish ($n\,{=}\,24$)]
      \plotpredictedactual{../results/predictions/embedding=contextual_model_name=TurkuNLP-bert-large-finnish-cased-v1_language=fi_window=1_operation=sum_similarity=cosine.csv}{Paired-B}
      \node[anchor = north west] at (rel axis cs: 0.05, 0.95) {$r = 0.679$};
      \nextgroupplot[title = Croatian ($n\,{=}\,112$)]
      \plotpredictedactual{../results/predictions/embedding=pooled_model_name=EMBEDDIA-crosloengual-bert_language=hr_window=3_operation=sum_similarity=cosine.csv}{Paired-C}
      \node[anchor = north west] at (rel axis cs: 0.05, 0.95) {$r = 0.717$};
      \nextgroupplot[title = Slovene ($n\,{=}\,111$)]
      \plotpredictedactual{../results/predictions/embedding=contextual_model_name=bert-base-multilingual-cased_language=sl_window=3_operation=sum_similarity=cosine.csv}{Paired-D}
      \node[anchor = north west] at (rel axis cs: 0.05, 0.95) {$r = 0.572$};
    \end{groupplot}
  \end{tikzpicture}
  \caption{The predicted and actual human judgments of the change in similarity of the best
    models for each language on the evaluation dataset.
    The best models are highlighted in \cref{table:evaluation-best-score}.
    The zero-mean Pearson correlation coefficient, i.e., the score, is given in the
    top-left corner of each plot (\cref{sec:task-definition}).
  }
  \label{chart:predicted}
\end{figure}


\newcommand{\plotscoretime}[2]{
  \addplot[
    only marks,
    mark size = 0.8pt,
    color = #2,
  ] table [
      col sep=comma,
      x = time,
      y = score,
    ]{#1};
}

\begin{figure}
  \centering
  \begin{tikzpicture}
    \begin{groupplot}[
        group style = {
            group size = 1 by 1,
            ylabels at = edge left,
            xlabels at = edge bottom,
          },
        width = 6in,
        height = 3in,
        xlabel = Time per instance (s),
        ylabel = Score,
        cycle list/Paired,
        legend pos = south east,
        legend style = {
            draw = none,
            font = \small,
          },
        legend cell align={left},
        x tick label style={
            /pgf/number format/.cd,
            fixed,
            fixed zerofill,
            precision=2,
            /tikz/.cd
          }
      ]
      \nextgroupplot
      \plotscoretime{../results/evaluation/model=static_language=en_window=0-50_operation=sum_similarity=cosine_time.csv}{Paired-A}
      \addlegendentry{static}
      \plotscoretime{../results/evaluation/model=contextual_language=en_window=0-10_operation=sum_similarity=cosine_time.csv}{Paired-B}
      \addlegendentry{contextual}
      \plotscoretime{../results/evaluation/model=pooled_language=en_window=0-10_operation=sum_similarity=cosine_time.csv}{Paired-C}
      \addlegendentry{pooled}
      \plotscoretime{../results/evaluation/model=static_language=fi_window=0-50_operation=sum_similarity=cosine_time.csv}{Paired-A}
      \plotscoretime{../results/evaluation/model=static_language=hr_window=0-50_operation=sum_similarity=cosine_time.csv}{Paired-A}
      \plotscoretime{../results/evaluation/model=static_language=sl_window=0-50_operation=sum_similarity=cosine_time.csv}{Paired-A}
      \plotscoretime{../results/evaluation/model=contextual_language=fi_window=0-10_operation=sum_similarity=cosine_time.csv}{Paired-B}
      \plotscoretime{../results/evaluation/model=contextual_language=hr_window=0-10_operation=sum_similarity=cosine_time.csv}{Paired-B}
      \plotscoretime{../results/evaluation/model=contextual_language=sl_window=0-10_operation=sum_similarity=cosine_time.csv}{Paired-B}
      \plotscoretime{../results/evaluation/model=pooled_language=fi_window=0-10_operation=sum_similarity=cosine_time.csv}{Paired-C}
      \plotscoretime{../results/evaluation/model=pooled_language=hr_window=0-10_operation=sum_similarity=cosine_time.csv}{Paired-C}
      \plotscoretime{../results/evaluation/model=pooled_language=sl_window=0-10_operation=sum_similarity=cosine_time.csv}{Paired-C}
    \end{groupplot}
  \end{tikzpicture}
  \caption{The scores on the evaluation dataset against the approximate time per
    instance, i.e., the total time divided by the number of instances, with the
    composition operation of addition.
  }
  \label{chart:score-time}
\end{figure}


In the main, greater scores were achieved with contextual and pooled embeddings than
with static embeddings (\cref{table:practice-best-score,table:evaluation-best-score}).
However, static embeddings make up a small fraction of the size of a contextual
language model.
For example, BERT's vocabulary size is approximately $30000$, the dimensions of the
\texttt{bert-base} and \texttt{bert-large} variants' hidden-states are $768$ and
$1024$, and their total parameters are $110$M and $340$M respectively
\parencites[4173-4174]{Devlin2019}.
Static embeddings thus make up approximately $21$\% and $9$\% of the total parameters.
It is also much faster to compute a contextualized representation from static
embeddings than to run inference on a language model.
For a naïve implementation of the procedure described in \cref{sec:methodology}, the
approximate time taken to compute the change in similarity between two words in context
is shown in \cref{chart:score-time}.
It is notably greater for contextual embeddings.
The right-most cluster is due to the \texttt{large} model variants.

\pgfplotstableset{
  col sep=comma,
  column type={l},
  columns={
      model,
      model_name,
      language,
      window,
      operation,
      similarity,
      score,
      time},
  columns/model/.style={
      column name=Embeddings,
      string type},
  columns/model_name/.style={
      column name=Model name,
      column type={L{2.75in}},
      string type,
      postproc cell content/.append style={
          @cell content={\texttt{##1}}}},
  columns/language/.style={
      column name=Language,
      string type,
      postproc cell content/.append style={
          @cell content={\texttt{##1}}}},
  columns/window/.style={
      column name=Window size,
      column type={R{0.85in}},
      int detect},
  columns/operation/.style={
      column name=Operation,
      string type},
  columns/similarity/.style={
      column name=Similarity measure,
      string type},
  columns/score/.style={
      column name=Score,
      column type={R{0.35in}},
      fixed,
      fixed zerofill,
      precision=3},
  columns/time/.style={
      column name=Time (s),
      fixed,
      fixed zerofill,
      precision=3},
  every head row/.style={before row=\toprule, after row=\midrule},
  every last row/.style={after row=\bottomrule},
  best/.style={@cell content=\underline{#1}},
}

\def\modelnameen{
  bert-base-multilingual-cased,
  bert-base-multilingual-uncased,
  EMBEDDIA/crosloengual-bert,
  bert-base-cased,
  bert-base-uncased,
  bert-large-cased,
  bert-large-uncased,
  bert-large-cased-whole-word-masking,
  bert-large-uncased-whole-word-masking,
}

\def\modelnamefi{
  bert-base-multilingual-cased,
  bert-base-multilingual-uncased,
  EMBEDDIA/crosloengual-bert,
  TurkuNLP/bert-base-finnish-cased-v1,
  TurkuNLP/bert-base-finnish-uncased-v1,
  TurkuNLP/bert-large-finnish-cased-v1,
}

\def\modelnamehr{
  bert-base-multilingual-cased,
  bert-base-multilingual-uncased,
  EMBEDDIA/crosloengual-bert,
  classla/bcms-bertic,
}

\def\modelnamesl{
  bert-base-multilingual-cased,
  bert-base-multilingual-uncased,
  EMBEDDIA/crosloengual-bert,
}

\pgfplotstableset{
  columns/model_name/.style={
      column name=Model name,
      column type={L{3in}},
      string type,
      postproc cell content/.append style={
          @cell content={\texttt{##1}}}},
}

\pgfplotsset{
  underline/.style={
      postproc cell content/.append code={
          \pgfkeysgetvalue{/pgfplots/table/@cell content}{\besttemp}
          \pgfkeysalso{best/.expand once={\besttemp}}
        }
    }
}

\begin{table}
  \centering
  \begin{subfigure}[b]{\textwidth}
    \centering
    \pgfplotstabletypeset[columns={
          language,
          model_name,
          window,
          score}]{../results/best/evaluation_best_static.csv}
    \caption{Static}
  \end{subfigure}
  \subfigurespace
  \begin{subfigure}[b]{\textwidth}
    \centering
    \pgfplotstabletypeset[columns={
          language,
          model_name,
          window,
          score},
      every row 0 column 3/.append style={
          postproc cell content/.append code={
              \pgfkeysgetvalue{/pgfplots/table/@cell content}{\besttemp}
              \pgfkeysalso{best/.expand once={\besttemp}}}},
      every row 1 column 3/.append style={
          postproc cell content/.append code={
              \pgfkeysgetvalue{/pgfplots/table/@cell content}{\besttemp}
              \pgfkeysalso{best/.expand once={\besttemp}}}},
      every row 3 column 3/.append style={
          postproc cell content/.append code={
              \pgfkeysgetvalue{/pgfplots/table/@cell content}{\besttemp}
              \pgfkeysalso{best/.expand once={\besttemp}}}}]{../results/best/evaluation_best_contextual.csv}
    \caption{Contextual}
  \end{subfigure}
  \subfigurespace
  \begin{subfigure}[b]{\textwidth}
    \centering
    \pgfplotstabletypeset[columns={
          language,
          model_name,
          window,
          score},
      every row 2 column 3/.append style={
          postproc cell content/.append code={
              \pgfkeysgetvalue{/pgfplots/table/@cell content}{\besttemp}
              \pgfkeysalso{best/.expand once={\besttemp}}}}]{../results/best/evaluation_best_pooled.csv}
    \caption{Pooled}
  \end{subfigure}
  \caption{The best scores on the evaluation dataset for each kind of embedding -- in
    all cases, it was obtained with the composition operation of addition.
    The best overall score for each language is underlined.
    The predicted and actual human judgments of the change in similarity for the best
    overall models are shown in \cref{chart:predicted}.
  }
  \label{table:evaluation-best-score}
\end{table}


I quantified the significance of the differences between the scores obtained with
different kinds of embeddings by paired $t$-tests and the Nemenyi test
\parencites{Demsar2006} on the scores of the best models over ten random samples of
90\% of the evaluation dataset (\cref{table:evaluation-best-score}).
For each pair of models, the null hypothesis was that the differences between the mean
scores were due to chance.
For each language, either contextual or pooled embeddings significantly outperformed
static embeddings, but did not differ significantly from each other
(\cref{table:significance}).
Hence, the results support the conclusion that contextual embeddings are more effective
than static embeddings for this task.

\pgfplotstableset{columns={
      embedding1,
      score1,
      embedding2,
      score2,
      tstatistic,
      pvalue,
      significant},
  columns/embedding1/.style={
      column name=Embeddings,
      string type},
  columns/score1/.style={
      column name={Score},
      column type={r},
      fixed,
      fixed zerofill,
      precision=3},
  columns/embedding2/.style={
      column name=Embeddings,
      string type},
  columns/score2/.style={
      column name={Score},
      column type={r},
      fixed,
      fixed zerofill,
      precision=3},
  columns/tstatistic/.style={
      column name={$t$-statistic},
      column type={R{0.65in}},
      relative*=1},
  columns/pvalue/.style={
      column name={$p$-value},
      column type={R{0.75in}},
      sci zerofill},
  columns/significant/.style={
      column name={$p$ < 0.05},
      column type={c},
      assign cell content/.code={
          \ifnum\pdfstrcmp{##1}{True}=0
            \pgfkeyssetvalue{/pgfplots/table/@cell content}{\checkmark}
          \else
            \pgfkeyssetvalue{/pgfplots/table/@cell content}{}
          \fi
        }
    },
  every head row/.append style={before row={
          \toprule
          \multicolumn{2}{c}{Model 1} & \multicolumn{2}{c}{Model 2} & & & \\
          \cmidrule(lr){1-2} \cmidrule(lr){3-4}
        }}}

\begin{table}[h!]
  \centering
  \begin{subfigure}{\textwidth}
    \centering
    \pgfplotstabletypeset{../results/cv/cv_test_results_nemenyi_en.csv}
    \caption{English ($n = 340$)}
  \end{subfigure}
  \subfigurespace
  \begin{subfigure}{\textwidth}
    \centering
    \pgfplotstabletypeset{../results/cv/cv_test_results_nemenyi_fi.csv}
    \caption{Finnish ($n = 24$)}
  \end{subfigure}
  \subfigurespace
  \begin{subfigure}{\textwidth}
    \centering
    \pgfplotstabletypeset{../results/cv/cv_test_results_nemenyi_hr.csv}
    \caption{Croatian ($n = 112$)}
  \end{subfigure}
  \subfigurespace
  \begin{subfigure}{\textwidth}
    \centering
    \pgfplotstabletypeset{../results/cv/cv_test_results_nemenyi_sl.csv}
    \caption{Slovene ($n = 111$)}
  \end{subfigure}
  \caption{The $t$-statistics from paired $t$-tests, and $p$-values from the Nemenyi
    test, on the scores obtained by the best models for each language and kind of
    embedding over ten random samples of 90\% of the evaluation dataset.
    The best models are highlighted in \cref{table:evaluation-best-score}.
    A positive $t$-statistic indicates that the mean score of `Model 1' is greater than
    that of `Model 2'.
  }
  \label{table:significance}
\end{table}


\subsection{Language-specificity of window-size dependence}
\label{sec:language-specificity}

Generally, I found that the scores obtained by all three types of embeddings were
maximized by a non-zero context-window size
(\cref{table:practice-best-score,table:evaluation-best-score}).
The influence of the window size is intuitive in the case of static embeddings.
Without a context window, the representations of a target word only differ between
contexts if the word is represented by different sub-word tokens in the different
contexts.
A similar argument applies to contextual embeddings, in that a target word may be
represented by multiple sub-word tokens that differ between contexts.
For the composition operation of addition, the scores against window size for each
language, kind of embedding, and model are shown in
\cref{chart:score-window-static,chart:score-window-contextual,chart:score-window-pooled}.

\newcommand{\plotscorewindow}[2]{
  \foreach \modelname in #2
    {
      \addplot+[
        sharp plot,
        mark = none,
        line width = 0.8pt,
        discard if not = {model_name}{\modelname},
      ] table [
          col sep=comma,
          x = window,
          y = score,
        ]{#1};
      \addlegendentryexpanded{\expandafter\texttt\expandafter{\modelname}}
    }
}

\begin{figure}
  \centering
  \begin{tikzpicture}
    \begin{groupplot}[
        group style = {
            group size = 2 by 2,
            ylabels at = edge left,
            vertical sep = 2.5in,
            horizontal sep = 0.7in,
          },
        width = 3in,
        height = 3in,
        xlabel = Window size,
        ylabel = Score,
        xtick = {0,10,...,50},
        ytick = {-0.2,-0.1,...,1.0},
        cycle list/Paired,
        legend style = {
            draw = none,
            font = \small,
            anchor = north,
            at = {(0.5,-0.2)},
          },
        legend cell align={left},
      ]
      \nextgroupplot[title = English ($n\,{=}\,340$)]
      \plotscorewindow{../results/evaluation/model=static_language=en_window=0-50_operation=sum_similarity=cosine.csv}{\modelnameen}
      \nextgroupplot[title = Finnish ($n\,{=}\,24$)]
      \plotscorewindow{../results/evaluation/model=static_language=fi_window=0-50_operation=sum_similarity=cosine.csv}{\modelnamefi}
      \nextgroupplot[title = Croatian ($n\,{=}\,112$)]
      \plotscorewindow{../results/evaluation/model=static_language=hr_window=0-50_operation=sum_similarity=cosine.csv}{\modelnamehr}
      \nextgroupplot[title = Slovene ($n\,{=}\,111$)]
      \plotscorewindow{../results/evaluation/model=static_language=sl_window=0-50_operation=sum_similarity=cosine.csv}{\modelnamesl}
    \end{groupplot}
  \end{tikzpicture}
  \caption{The score on the evaluation dataset against window size for static embedding
    models and the composition operation of addition.}
  \label{chart:score-window-static}
\end{figure}

\begin{figure}
  \centering
  \begin{tikzpicture}
    \begin{groupplot}[
        group style = {
            group size = 2 by 2,
            ylabels at = edge left,
            vertical sep = 2.5in,
            horizontal sep = 0.7in,
          },
        width = 3in,
        height = 3in,
        xlabel = Window size,
        ylabel = Score,
        xtick = {0,1,...,10},
        ytick = {-0.2,-0.1,...,1.0},
        cycle list/Paired,
        legend style = {
            draw = none,
            font = \small,
            anchor = north,
            at = {(0.5,-0.2)},
          },
        legend cell align={left},
      ]
      \nextgroupplot[title = English ($n\,{=}\,340$)]
      \plotscorewindow{../results/evaluation/model=contextual_language=en_window=0-10_operation=sum_similarity=cosine.csv}{\modelnameen}
      \nextgroupplot[title = Finnish ($n\,{=}\,24$)]
      \plotscorewindow{../results/evaluation/model=contextual_language=fi_window=0-10_operation=sum_similarity=cosine.csv}{\modelnamefi}
      \nextgroupplot[title = Croatian ($n\,{=}\,112$)]
      \plotscorewindow{../results/evaluation/model=contextual_language=hr_window=0-10_operation=sum_similarity=cosine.csv}{\modelnamehr}
      \nextgroupplot[title = Slovene ($n\,{=}\,111$)]
      \plotscorewindow{../results/evaluation/model=contextual_language=sl_window=0-10_operation=sum_similarity=cosine.csv}{\modelnamesl}
    \end{groupplot}
  \end{tikzpicture}
  \caption{The score on the evaluation dataset against window size for contextual
    embedding models and the composition operation of addition.}
  \label{chart:score-window-contextual}
\end{figure}

\begin{figure}
  \centering
  \begin{tikzpicture}
    \begin{groupplot}[
        group style = {
            group size = 2 by 2,
            ylabels at = edge left,
            vertical sep = 2.5in,
            horizontal sep = 0.7in,
          },
        width = 3in,
        height = 3in,
        xlabel = Window size,
        ylabel = Score,
        xtick = {0,1,...,10},
        ytick = {-0.2,-0.1,...,1.0},
        cycle list/Paired,
        legend style = {
            draw = none,
            font = \small,
            anchor = north,
            at = {(0.5,-0.2)},
          },
        legend cell align={left},
      ]
      \nextgroupplot[title = {English ($n\,{=}\,340$)}]
      \plotscorewindow{../results/evaluation/model=pooled_language=en_window=0-10_operation=sum_similarity=cosine.csv}{\modelnameen}
      \nextgroupplot[title = {Finnish ($n\,{=}\,24$)}]
      \plotscorewindow{../results/evaluation/model=pooled_language=fi_window=0-10_operation=sum_similarity=cosine.csv}{\modelnamefi}
      \nextgroupplot[title = {Croatian ($n\,{=}\,112$)}]
      \plotscorewindow{../results/evaluation/model=pooled_language=hr_window=0-10_operation=sum_similarity=cosine.csv}{\modelnamehr}
      \nextgroupplot[title = {Slovene ($n\,{=}\,111$)}]
      \plotscorewindow{../results/evaluation/model=pooled_language=sl_window=0-10_operation=sum_similarity=cosine.csv}{\modelnamesl}
    \end{groupplot}
  \end{tikzpicture}
  \caption{The score on the evaluation dataset against window size for pooled embedding
    models and the composition operation of addition.}
  \label{chart:score-window-pooled}
\end{figure}


\textcites[3]{Virtanen2019} have noted that, for a random sample of 1\% of the relevant
Wikipedia dataset, the number of sub-word tokens that represent a word
is greater for Finnish ($1.97$) than for English ($1.16$) with the multilingual BERT model.
This is attributed to the morphological complexity of Finnish and its comparatively
small fraction of the model's vocabulary.
Accordingly, I found that Finnish-specific models generally outperformed multilingual
ones and that the scores varied more widely with window size for Finnish than the other
languages.
