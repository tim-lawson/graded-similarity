\section{Conclusion}
\label{sec:conclusion}

In this paper, I have presented the results of a hypothetical submission to
SemEval-2020 Task 3, ``Graded Word Similarity in Context''.
The purpose of this investigation was to compare the performance of static and
contextual embeddings and their composition within a fixed-size context window on the
task of predicting the change in the human judgment of similarity of a pair of words in
two different contexts.
I found that contextual embeddings generally outperformed static embeddings, but at a
significant computational cost, and that composition benefited both static and
contextual embeddings of sub-word tokens, with highly language-specific dependence on
the window size.

The results that I have given must be interpreted in context: the original submission
authors did not have access to the evaluation dataset prior to submitting their
results, only the `practice kit' of very few instances, and were therefore unable to
optimise a parameter such as the window size prior to submission.
Additionally, the models that I used were not necessarily available to the authors.
With these caveats, I achieved several notable results (\cref{table:best-score}):
\begin{itemize}
  \item The \texttt{pooled} variant of \texttt{EMBEDDIA/crosloengual-bert} with a window
        size of three would have placed second among the Croatian submissions, with a
        score of $0.717$.
  \item The \texttt{contextual} variant of \texttt{TurkuNLP/bert-base-finnish-uncased-v1}
        with a window size of zero would have placed fourth among the Finnish
        submissions, with a score of $0.679$.
  \item The \texttt{static} variant of \texttt{TurkuNLP/bert-base-finnish-uncased-v1}
        with a window size of zero outperform several of the Finnish submissions,
        including the baseline, with a score of $0.564$.
\end{itemize}
Given the significant expense of applying contextual-embedding models, these results
highlight the importance of analysing the complexity of the task at hand and
considering the possibility that a simpler model produces adequate results.
