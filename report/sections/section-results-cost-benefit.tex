\subsection{Cost-benefit analysis of contextual embeddings}
\label{sec:cost-benefit}

\begin{figure}
  \centering
  \newcommand{\plotpredictedactual}[2]{
  \addplot[
    only marks,
    mark size = 0.8pt,
    color = #2,
  ] table [
      col sep=comma,
      x = predicted,
      y = actual,
    ]{#1};
}

\begin{tikzpicture}
  \begin{groupplot}[
      group style = {
          group size = 2 by 2,
          ylabels at = edge left,
          xlabels at = edge bottom,
          vertical sep = 0.7in,
          horizontal sep = 0.5in,
        },
      scale only axis,
      width = 2.5in,
      height = 1.5in,
      xlabel = Predicted,
      ylabel = Actual,
      cycle list/Paired,
      legend pos = south east,
      legend style = {
          draw = none,
          font = \small,
        },
      legend cell align={left},
    ]
    \nextgroupplot[title = English ($n\,{=}\,340$)]
    \plotpredictedactual{../results/predictions/embedding=contextual_model_name=bert-base-uncased_language=en_window=1_operation=sum_similarity=cosine.csv}{Paired-A}
    \node[anchor = north west] at (rel axis cs: 0.05, 0.95) {$r = 0.660$};
    \nextgroupplot[title = Finnish ($n\,{=}\,24$)]
    \plotpredictedactual{../results/predictions/embedding=contextual_model_name=TurkuNLP-bert-large-finnish-cased-v1_language=fi_window=1_operation=sum_similarity=cosine.csv}{Paired-B}
    \node[anchor = north west] at (rel axis cs: 0.05, 0.95) {$r = 0.679$};
    \nextgroupplot[title = Croatian ($n\,{=}\,112$)]
    \plotpredictedactual{../results/predictions/embedding=pooled_model_name=EMBEDDIA-crosloengual-bert_language=hr_window=3_operation=sum_similarity=cosine.csv}{Paired-C}
    \node[anchor = north west] at (rel axis cs: 0.05, 0.95) {$r = 0.717$};
    \nextgroupplot[title = Slovene ($n\,{=}\,111$)]
    \plotpredictedactual{../results/predictions/embedding=contextual_model_name=bert-base-multilingual-cased_language=sl_window=3_operation=sum_similarity=cosine.csv}{Paired-D}
    \node[anchor = north west] at (rel axis cs: 0.05, 0.95) {$r = 0.572$};
  \end{groupplot}
\end{tikzpicture}
  \caption{The predicted and actual human judgments of the change in similarity of the best
    models for each language on the evaluation dataset.
    The best models are highlighted in \cref{table:best-score-evaluation}.
    The zero-mean Pearson correlation coefficient (\cref{sec:task-definition}), i.e., the
    score, is given in the top-left corner of each plot.
  }
  \label{figure:predicted-actual}
\end{figure}

\begin{figure}
  \centering
  \newcommand{\plotscoretime}[2]{
  \addplot[
    color=#2,
    mark size=0.8pt,
    only marks
  ] table [
      col sep=comma,
      x=time,
      y=score
    ]{#1};
}

\begin{tikzpicture}
  \begin{groupplot}[
      cycle list/Paired,
      group style={
          group size=1 by 1,
          xlabels at=edge bottom,
          ylabels at=edge left
        },
      height=2in,
      legend cell align={left},
      legend pos=south east,
      legend style={
          draw=none,
          font=\small
        },
      scale only axis,
      width=5.5in,
      x tick label style={
          /pgf/number format/.cd,
          fixed,
          fixed zerofill,
          precision=2,
          /tikz/.cd
        },
      xlabel=Time per instance (s),
      ylabel=Score
    ]
    \nextgroupplot
    \plotscoretime{../results/evaluation/model=static_language=en_window=0-50_operation=sum_similarity=cosine_time.csv}{Paired-A}
    \addlegendentry{static}
    \plotscoretime{../results/evaluation/model=contextual_language=en_window=0-10_operation=sum_similarity=cosine_time.csv}{Paired-B}
    \addlegendentry{contextual}
    \plotscoretime{../results/evaluation/model=pooled_language=en_window=0-10_operation=sum_similarity=cosine_time.csv}{Paired-C}
    \addlegendentry{pooled}
    \plotscoretime{../results/evaluation/model=static_language=fi_window=0-50_operation=sum_similarity=cosine_time.csv}{Paired-A}
    \plotscoretime{../results/evaluation/model=static_language=hr_window=0-50_operation=sum_similarity=cosine_time.csv}{Paired-A}
    \plotscoretime{../results/evaluation/model=static_language=sl_window=0-50_operation=sum_similarity=cosine_time.csv}{Paired-A}
    \plotscoretime{../results/evaluation/model=contextual_language=fi_window=0-10_operation=sum_similarity=cosine_time.csv}{Paired-B}
    \plotscoretime{../results/evaluation/model=contextual_language=hr_window=0-10_operation=sum_similarity=cosine_time.csv}{Paired-B}
    \plotscoretime{../results/evaluation/model=contextual_language=sl_window=0-10_operation=sum_similarity=cosine_time.csv}{Paired-B}
    \plotscoretime{../results/evaluation/model=pooled_language=fi_window=0-10_operation=sum_similarity=cosine_time.csv}{Paired-C}
    \plotscoretime{../results/evaluation/model=pooled_language=hr_window=0-10_operation=sum_similarity=cosine_time.csv}{Paired-C}
    \plotscoretime{../results/evaluation/model=pooled_language=sl_window=0-10_operation=sum_similarity=cosine_time.csv}{Paired-C}
  \end{groupplot}
\end{tikzpicture}

  \caption{The scores on the evaluation dataset against the approximate time per
    instance, i.e., the total time divided by the number of instances, with the
    composition operation of addition.
  }
  \label{figure:score-time-evaluation}
\end{figure}

In the main, greater scores were achieved with contextual and pooled embeddings than
with static embeddings (\cref{table:practice-best-score,table:best-score-evaluation}).
However, static embeddings make up a small fraction of the size of a contextual
language model.
For example, BERT's vocabulary size is approximately $30000$, the dimensions of the
\texttt{bert-base} and \texttt{bert-large} variants' hidden-states are $768$ and
$1024$, and their total parameters are $110$M and $340$M respectively
\parencites[4173-4174]{Devlin2019}.
Static embeddings thus make up approximately $21$\% and $9$\% of the total parameters.
It is also much faster to compute a contextualized representation from static
embeddings than to run inference on a language model.
For a naïve implementation of the procedure described in \cref{sec:methodology}, the
approximate time taken to compute the change in similarity between two words in context
is shown in \cref{figure:score-time-evaluation}.
It is notably greater for contextual embeddings, and the right-most cluster is due to
the \texttt{large} model variants.

\begin{table}
  \centering
  \begin{subfigure}[b]{\textwidth}
    \centering
    \pgfplotstabletypeset[
  columns={
      language,
      model_name,
      window,
      score
    }
]{../results/best/evaluation_best_static.csv}
    \caption{Static}
  \end{subfigure}
  \subfigurespace
  \begin{subfigure}[b]{\textwidth}
    \centering
    \pgfplotstabletypeset[columns={
      language,
      model_name,
      window,
      score},
  every row 0 column 3/.append style={
      postproc cell content/.append code={
          \pgfkeysgetvalue{/pgfplots/table/@cell content}{\besttemp}
          \pgfkeysalso{best/.expand once={\besttemp}}}},
  every row 1 column 3/.append style={
      postproc cell content/.append code={
          \pgfkeysgetvalue{/pgfplots/table/@cell content}{\besttemp}
          \pgfkeysalso{best/.expand once={\besttemp}}}},
  every row 3 column 3/.append style={
      postproc cell content/.append code={
          \pgfkeysgetvalue{/pgfplots/table/@cell content}{\besttemp}
          \pgfkeysalso{best/.expand once={\besttemp}}}}]{../results/best/evaluation_best_contextual.csv}
    \caption{Contextual}
  \end{subfigure}
  \subfigurespace
  \begin{subfigure}[b]{\textwidth}
    \centering
    \pgfplotstabletypeset[
  columns={
      language,
      model_name,
      window,
      score
    },
  every row 2 column 3/.append style={
      postproc cell content/.append code={
          \pgfkeysgetvalue{/pgfplots/table/@cell content}{\besttemp}
          \pgfkeysalso{best/.expand once={\besttemp}
          }
        }
    }
]{../results/best/evaluation_best_pooled.csv}
    \caption{Pooled}
  \end{subfigure}
  \caption{The best scores on the evaluation dataset for each kind of embedding -- in
    all cases, it was obtained with the composition operation of addition.
    The best overall score for each language is underlined.
    The predicted and actual human judgments of the change in similarity for the models
    with the best overall scores are shown in \cref{figure:predicted-actual}.
  }
  \label{table:best-score-evaluation}
\end{table}

I quantified the significance of the differences between the scores obtained with
different kinds of embeddings by paired $t$-tests and the Nemenyi test
\parencites{Demsar2006} on the scores of the best models over ten random samples of
90\% of the evaluation dataset (\cref{table:best-score-evaluation}).
For each pair of models, the null hypothesis was that the differences between the mean
scores were due to chance.
For each language, either contextual or pooled embeddings significantly outperformed
static embeddings, but did not differ significantly from each other
(\cref{table:significance}).
Hence, the results support the conclusion that contextual embeddings are more effective
than static embeddings for this task.

\pgfplotstableset{columns={
      embedding1,
      score1,
      embedding2,
      score2,
      tstatistic,
      pvalue,
      significant},
  columns/embedding1/.style={
      column name=Embeddings,
      string type},
  columns/score1/.style={
      column name={Score},
      column type={r},
      fixed,
      fixed zerofill,
      precision=3},
  columns/embedding2/.style={
      column name=Embeddings,
      string type},
  columns/score2/.style={
      column name={Score},
      column type={r},
      fixed,
      fixed zerofill,
      precision=3},
  columns/tstatistic/.style={
      column name={$t$-statistic},
      column type={R{0.65in}},
      relative*=1},
  columns/pvalue/.style={
      column name={$p$-value},
      column type={R{0.75in}},
      sci zerofill},
  columns/significant/.style={
      column name={$p$ < 0.05},
      column type={c},
      assign cell content/.code={
          \ifnum\pdfstrcmp{##1}{True}=0
            \pgfkeyssetvalue{/pgfplots/table/@cell content}{\checkmark}
          \else
            \pgfkeyssetvalue{/pgfplots/table/@cell content}{}
          \fi
        }
    },
  every head row/.append style={before row={
          \toprule
          \multicolumn{2}{c}{Model 1} & \multicolumn{2}{c}{Model 2} & & & \\
          \cmidrule(lr){1-2} \cmidrule(lr){3-4}
        }}}

\begin{table}[h!]
  \centering
  \begin{subfigure}{\textwidth}
    \centering
    \pgfplotstabletypeset{../results/cv/cv_test_results_nemenyi_en.csv}
    \caption{English ($n = 340$)}
  \end{subfigure}
  \subfigurespace
  \begin{subfigure}{\textwidth}
    \centering
    \pgfplotstabletypeset{../results/cv/cv_test_results_nemenyi_fi.csv}
    \caption{Finnish ($n = 24$)}
  \end{subfigure}
  \subfigurespace
  \begin{subfigure}{\textwidth}
    \centering
    \pgfplotstabletypeset{../results/cv/cv_test_results_nemenyi_hr.csv}
    \caption{Croatian ($n = 112$)}
  \end{subfigure}
  \subfigurespace
  \begin{subfigure}{\textwidth}
    \centering
    \pgfplotstabletypeset{../results/cv/cv_test_results_nemenyi_sl.csv}
    \caption{Slovene ($n = 111$)}
  \end{subfigure}
  \caption{The $t$-statistics from paired $t$-tests, and $p$-values from the Nemenyi
    test, on the scores obtained by the best models for each language and kind of
    embedding over ten random samples of 90\% of the evaluation dataset.
    The best models are highlighted in \cref{table:evaluation-best-score}.
    A positive $t$-statistic indicates that the mean score of `Model 1' is greater than
    that of `Model 2'.
  }
  \label{table:significance}
\end{table}


\begin{table}[h!]
  \centering
  \begin{subfigure}{\textwidth}
    \centering
    \pgfplotstabletypeset{../results/cv/cv_test_results_nemenyi_en.csv}
    \caption{English ($n = 340$)}
  \end{subfigure}
  \subfigurespace
  \begin{subfigure}{\textwidth}
    \centering
    \pgfplotstabletypeset{../results/cv/cv_test_results_nemenyi_fi.csv}
    \caption{Finnish ($n = 24$)}
  \end{subfigure}
  \subfigurespace
  \begin{subfigure}{\textwidth}
    \centering
    \pgfplotstabletypeset{../results/cv/cv_test_results_nemenyi_hr.csv}
    \caption{Croatian ($n = 112$)}
  \end{subfigure}
  \subfigurespace
  \begin{subfigure}{\textwidth}
    \centering
    \pgfplotstabletypeset{../results/cv/cv_test_results_nemenyi_sl.csv}
    \caption{Slovene ($n = 111$)}
  \end{subfigure}
  \caption{The $t$-statistics from paired $t$-tests, and $p$-values from the Nemenyi
    test, on the scores obtained by the best models for each language and kind of
    embedding over ten random samples of 90\% of the evaluation dataset.
    The best models are highlighted in \cref{table:best-score-evaluation}.
    A positive $t$-statistic indicates that the mean score of `Model 1' is greater than
    that of `Model 2'.
  }
  \label{table:significance}
\end{table}
